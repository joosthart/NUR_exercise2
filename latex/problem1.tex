
\section{Dark Matter Halo}
In this section, the results of exercise 1 will be discussed. In this exercise, initial conditions are made for a dark matter halo of an isolated galaxy. This is done using the Hernquist profile which is given by,
\begin{equation}\label{eq:her}
  \rho(r) = \frac{M_\mathrm{dm}}{2\pi}\frac{a}{r(r+a)^3}.
\end{equation}
We assume here $M_\mathrm{dm} = 10^{12}$ and $a=80\ \mathrm{kpc}$.

\subsection*{1a}\label{sec:1a}
The goal of this exercise is to write a random number generator. As a random number generator (RNG), the output a linear congruential generator is used as the input of an two XOR-shift differnt generator, which exclusice or is used for another linear congruential generator ($\mathrm{LCG}_1 \rightarrow (\mathrm{XOR}_1 \oplus  \mathrm{XOR}_2)  \rightarrow \mathrm{LCG}_2$). The values of the parameters ar listed in table \ref{tab:rand}.

\begin{table}[]
  \centering
  \caption{Parameters used for random number generators.}
  \label{tab:rand}
  \begin{tabular}{c|l}
    Random Number Generator & \multicolumn{1}{c}{Parameters}                                                               \\ \hline
    $\mathrm{LGC}_1$        & \begin{tabular}[c]{@{}l@{}}$a =1664525$ \\ $b = 1013904223$\end{tabular}                     \\ \hline
    $\mathrm{LGC}_2$        & \begin{tabular}[c]{@{}l@{}}$a =6364136223846793005$\\ $b = 1442695040888963407$\end{tabular} \\ \hline
    $\mathrm{XOR}_1$        & $a=19, b=25, c=7$  \\ \hline
    $\mathrm{XOR}_2$        & $a=15, b=13, c=9$                                                                
  \end{tabular}
\end{table}

The RNG is tested in three ways:
\begin{enumerate}
  \item A plot has been generated showing the first 1000 points $x_{i+1}$ vs $x_i$. This plot is shown in figure \ref{fig:1a1}. The point do not show any structure, as far as the eye can see.
  \item 1,000,000 points are generated using the RNG. These points are shown as a histogram in figure \ref{fig:1a2}
  \item At last, the Pearson correlation coefficient are calculated for $x_i,x_{i+1}$ and $x_i, x_{i+2}$, using 100,000 randomly generated numbers. The obtained values are: $r_{x_i,x_{i+1}} = $ \input{./output/1a_pearson_correlation1.txt} and $r_{x_i,x_{i+2}} = $ \input{./output/1a_pearson_correlation2.txt}. These values show that there is no significant correlation between sequential random numbers.
\end{enumerate}

\begin{figure}[ht]
  \centering
  \begin{subfigure}[t]{0.49\textwidth}
    
    \includegraphics[width=\linewidth]{./plots/1a_sequential_random_numbers.png}
    \caption{First 1000 random points plotted as $x_{i+1}$ vs $x_i$.}
    \label{fig:1a1}
  \end{subfigure}
  ~
  \begin{subfigure}[t]{0.49\textwidth}
    \includegraphics[width=\linewidth]{./plots/1a_histogram_random_numbers.png}
    \caption{Histogram of 1,000,000 points generated using RNG.}
    \label{fig:1a2}
  \end{subfigure}
  \caption{}
\end{figure}



\subsection*{1b}
In this exercise, $10^6$ particles are generated following the radial profile of the Hernquist density. The RNG from \ref{sec:1a} is used. The radius of the particles are sampled in spherical coordinates. In order to sample $r$ according to the density profile \ref{eq:her}, we first need to calculate the cumulative distribution faction for equation \ref{eq:her}. First, the enclosed mass fraction, of the Hernquist profile, is:
\begin{equation}
  m(<r) = \int_0^r \int_0^\pi \int_0^{2\pi} \rho(r') r'^2 \sin\phi \mathrm{d}\phi\mathrm{d}\theta\mathrm{d}r'=\frac{M_\mathrm{dm}r^2}{\left(a+r\right)^2},
\end{equation}
where $\rho$ is the Hernquist profile. The total mass in the Hernquist distribution is $M_\mathrm{dm}$. The cumulative density profile, $F$ can thus be written as
\begin{equation}\label{eq:theory}
  F(r) = \frac{r^2}{\left(a+r\right)^2}.
\end{equation}
To sample $F(r)$ using random numbers $u$, which are uniformly between 0 and 1, we need $F^{-1}(u) = r$, which is given by
\begin{equation}\label{eq:hern_r}
  F^{-1}(u) = r = \frac{a\sqrt{u}}{1-\sqrt{u}}.
\end{equation}
This equation is used to sample the Hernquist profile. The enclosed fraction particles of the sample is shown in figure \ref{fig:1b}. The theoretical enclosed fraction of particles c is given by equation \ref{eq:theory} and is also shown in the figure. In the figure can be seen that the sampled enclosed mass fraction is very similar to the theoretical enclosed mass fraction.

\begin{figure}[!ht]
  \centering
  \includegraphics[width=0.49\linewidth]{./plots/1b_enclosed_mass_fraction.png}
  \caption{Expected enclosed mass fraction and enclosed mass fraction of randomly generated points, as a function of radius.}
  \label{fig:1b}
\end{figure}

\subsection*{1c}
Using the RNG, here a 3D sample following the a Hernquist profile (eq. \ref{eq:her}) is generated. If points are uniformly sampled in $(r,\phi, \theta)$, the density of points will not be uniform inside the sphere. Let $U$ be uniform random points between 0 and 1. Than we can sample spherically uniform points in spherical coordinates with the following equations from tutorial 4:
\begin{eqnarray}
  \phi =& 2\pi U, \\
  \theta =& \arccos(1-2U)
\end{eqnarray}
These equations are used together with equation \ref{eq:hern_r} to sample the Hernquist profile. The results are shown in figure \ref{fig:1c1}.

Furthermore, a plot showing the distribution of the particles in the $(\phi,\theta)$-plane is given in figure \ref{fig:1c2}. In the $\theta$-direction, more points are closer to $\frac{\pi}{2}$ and less at $0$ and $\pi$. This is as expected.

\begin{figure}[!ht]
  \centering
  \includegraphics[width=0.49\linewidth]{./plots/1c_3d_scatter_hernquist.png}
  \caption{Sampled Hernquist profile for $10^3$ particles. The figure is not showning all particles. The sample containes several outliers at $r \sim 100\ \mathrm{Mpc}$. Most of the points are inside the boundaries of the figure.}
  \label{fig:1c1}
\end{figure}

\begin{figure}[!ht]
  \centering
  \includegraphics[width=0.49\linewidth]{./plots/1c_3d_scatter_phi_theta.png}
  \caption{Distribution of particles in $(\phi,\theta)$-plane of sample shown in figure \ref{fig:1c1}.}
  \label{fig:1c2}
\end{figure}

\subsection*{1d}
Here, the derivative of equation \ref{eq:her} at $r=1.2a$, is calculated numerically and analytically. The numeric calculations is performed using Ridders' method. The $m$ which yields the best approximation is 10. The obtained numerical, $\mathrm{d}(\hat{\rho}))/\mathrm{d}(r)$, and analytical, $\mathrm{d}(\rho)/\mathrm{d}(r)$ are \input{./output/1d_derivative_hernquist_hat.txt} and \input{./output/1d_derivative_hernquist.txt}, respectively.

\subsection*{1e}
The critical density is given by $\rho_c = 8.5 \times 10^{-27} \mathrm{kg/m}^3 \approx 150 M_\odot/\mathrm{kpc}^3$. $R_\Delta$ and $M_\Delta$ are the radius and mass for which $\Delta$ is the amount of time the critical density is exceeded. For our case, $R_\Delta$ is the radius at which $\rho_\mathrm{dm}(r) = \Delta\rho_c$. $R_{200}$ and $R_{500}$ are calculated using the bisection method on $\rho_\mathrm{dm}(r) - \Delta\rho_c$, where $\Delta = 200, 500$, respectively. The obtained values are: $R_{200} = $ \input{./output/1e_R200.txt}kpc and $R_{500} = $ \input{./output/1e_R500.txt}kpc.

\subsection*{1f}
At last, for this exercise, we will look at an asymmetric potential of a Hernquist profile in 2D. The potential is given by
\begin{equation}
  \Phi = - \frac{GM_\mathrm{dm}}{\sqrt{(x-1.3\mathrm{kpc})^2 + 2(y-4.2\mathrm{kpc})^2}+a}.
\end{equation}
The minimum of this potential is calculated using the Quasi-Newton method, starting from $(-1000 \mathrm{kpc}, -200\mathrm{kpc})$. As a linear solver the Golden Section algorithm is used. In this calculation $G$ is set to 1, since it is only a scaling factor, and does not affect the shape of the potential. In figure \ref{fig:1f}, the distance from the center (1.3, 4.2) is shown at every iteration.

\begin{figure}[!ht]
  \centering
  \includegraphics[width=0.49\linewidth]{./plots/1f_3d_distance_to_minimum.png}
  \caption{Distance at every iteration to (1.3, 4.2), of the Quasi-Newton algorithm.}
  \label{fig:1f}
\end{figure}


\pagebreak

\subsection*{Code}

\lstinputlisting{./code/problem1.py}

\newpage